%!TEX root = ../olga-cv.tex
%----------------------------------------------------------------------------------
%        Research Experience 2
%----------------------------------------------------------------------------------
	\cventry{2009-2014}
		{LLNL}{Lawrence Graduate Scholar}{}{}
		%{Developing a framework for asynchronous scalable correction of application load imbalance}
		{\begin{itemize}
            \item Developed a model for selection of load
                  balancing algorithms for a given simulation state~\cite{pearce:ics12}
            \item Developed a fast \& accurate load balance method for highly non-uniform density N-Body simulations~\cite{pearce:ics14}
            \item Developed a tool for decoupling \& offloading load balance computation,
                  enabling lazy load balancing~\cite{pearce:ipdps16}
			      %\item Developed an application-independent load model and metrics;
			%\item Developed an algorithmic methodology for deciding {\em when} and {\em how} to correct application imbalance;
			%\item Developed a fast and accurate method of assigning and load balancing interactions in N-body applications;
			%\item Developed a framework for decoupling load balance algorithm resources from the application;
			%\item Developed a framework for offloading \& overlapping load balance computation with application execution;
			%\item Applied work to several large physics applications at LLNL.
		\end{itemize}}

	\cventry{2004-2009}
		{Texas A\&M University, Department of Computer Science}{Research Assistant}{}{}
		{\begin{itemize}
			\item Developed distributed data structures, parallel algorithms,
				and adaptive algorithm selection methods~\cite{thomas:ppopp05}
				in the Standard Template Adaptive Parallel Library (STAPL),
			  	a parallel superset of C++ STL~\cite{tanase:ppopp11,buss:systor10,tanase:lcpc09}
              			%Implemented and optimized a parallel vector.
			\item Designed lock-free containers for multi-threaded programming
			%\item Working on a general scheduler and partitioner for STAPL, which will
			%be used to schedule and load-balance various applications using STAPL.
			\item Developed a Dynamic Graph Composition Library for updating graph metrics of large, dynamic graphs
		\end{itemize}}

	\cventry{May-Oct 2007,\\July-Aug 2008}
		{LLNL}{Research Intern}{}{}
		{\begin{itemize}
			\item Designed techniques to cheaply estimate processor loads and decide
				when to rebalance a simulation
			\item Designed load balancing techniques for multi-physics simulations,
				including multi-constraint domain repartitioning (data parallelism)
				and domain replication (task parallelism)
		\end{itemize}}

%	\cventry{Summer 2003,\\Summer 2004}
%		{\href{http://www.cse.tamu.edu/}{Texas A\&M University, Department of Computer Science}}{Undergraduate Research Intern}{}{}
%		{Designed and implemented a parallel array data structure and multiple
%		parallel algorithms for the Standard Template Adaptive Parallel Library
%		(STAPL), a parallel library for C++, superset of STL.}
