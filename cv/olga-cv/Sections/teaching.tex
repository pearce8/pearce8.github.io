%!TEX root = ../olga-cv.tex
%----------------------------------------------------------------------------------
%            Teaching
%----------------------------------------------------------------------------------
\section{Teaching and Mentoring}
	\cventry{2015-present}{Student Research Mentoring}{Lawrence Livermore National Laboratory}{}{}{
		\begin{itemize}
		\item (Fall 2019 - Present) Bengisu Elis, CS major (PhD), Techical University of Munich, Germany.
		    Working on research directions for MPI on heterogeneous architectures.
        \item (Summer 2020 - Present) Connor Scully-Allison, CS major (PhD), University of Arizona.
		  Working on optimization of \href{https://hatchet.readthedocs.io/en/latest/}{Hatchet},
          a call-tree based tool for performance analysis of HPC applications.
        \item (Spring 2020 - Present) Ian Lumsden, CS major (PhD), University of Tennessee.
		  Working on a query language for \href{https://hatchet.readthedocs.io/en/latest/}{Hatchet},
          a call-tree based tool for performance analysis of HPC applications.
        \item (Summer 2020) Kawthar Shafie Khorassani, CS major (PhD), The Ohio State University.
		  Worked on GPUDirect Async Designs with MVAPICH2-GDR.
        \item (Summer 2020) Katy Williams, CS major (PhD), University of Arizona.
		  Worked on interactive tree visualization for \href{https://hatchet.readthedocs.io/en/latest/}{Hatchet},
          a call-tree based tool for performance analysis of HPC applications.
		\item (Summer 2018 and 2019) David Eberius, CS major (PhD), University of Tennessee.
		  Worked on porting \href{http://www.exmatex.org/comd.html}{CoMD}, an ExMatEx proxy application to
		  \href{https://github.com/LLNL/RAJA}{RAJA}, and optimizing communication on GPUs.
		  Worked on load balancing metrics for heterogeneous architectures.
		\item (Summer 2018) Kewen Meng, CS major (PhD), University of Oregon.
		  Worked on porting \href{http://www.exmatex.org/comd.html}{CoMD}, an ExMatEx proxy application to
		  \href{https://github.com/LLNL/RAJA}{RAJA},
		  and optimizing performance for the CUDA RAJA backend.
		\item (Summer 2016) Hadia Ahmed, CS major (PhD), University of Alabama at Birmingham.
		  Studied performance impact of initial load imbalance and dynamic imbalance in the modified version of
		  \href{http://www.exmatex.org/comd.html}{CoMD}, an ExMatEx proxy application.
		\item (Summer 2015) Rasmus Larsen, CS major (MS), University of Copenhagen. %   \hfill {\em Summer 2015}\\
		  Enabled overdecomposition and dynamic work redistribution in
		  \href{http://www.exmatex.org/comd.html}{CoMD}, an ExMatEx proxy application.
		  Implemented a mechanism for introducing initial load imbalance and
		  varying the imbalance throughout the simulation
		  to facilitate exploration of various load balancing schemes.
		\end{itemize}
	}
	\cventry{2007}{Student Team Mentoring}{Cluster Challenge}{ACM/IEEE Supercomputing Conference (SC)}{}{
		\begin{itemize}
		\item Mentored teams participating in the challenge, provided technical expertise in parallel performance and
			parallel applications ported by the teams to their clusters during the challenge.
		\end{itemize}
		}
	\cventry{2005-2009}{Student Research Mentoring}{Texas A\&M University}{}{}{
		\begin{itemize}
		\item Jeremy Vu, CSE major, Texas A\&M University. %   \hfill {\em Spring 2009}\\
			Worked on parallel algorithms in STAPL.
		\item Harshvardhan, CS major, %\hfill {\em Fall 2005 -- Spring 2007}\\
			%University Undergraduate Research Fellow,
			Texas A\&M. %, graduated December 2008.
			Worked on STAPL Graph, Dynamic Graph Composition Library.
		\item Kelli Bacon, CRA-W DMP Program. CE major, Gonzaga U. %  \hfill {\em Summer 2006} \\
			Dynamic Graph Composition Library.
		\item Saransh Mittal, CSE major, %   \hfill {\em Summer 2006} \\
			Indian Inst. of Technology, Bombay.
			Dynamic Graph Composition Library.
		\item Anna Tikhonova, CRA-W DMP Program. CS major, U. of San Francisco. %   \hfill {\em Summer 2005} \\
			%Currently a Ph.D. student at University of California - Davis.\\
			Parallel algorithms in STAPL.
		\end{itemize}
		}
	\cventry{2006}
		{Fellow}{\href{http://gta.tamu.edu/}{Graduate Teaching Academy}}{Texas A\&M University}{}{}

	\cventry{2006}{Student Mentoring Award}{Aggie Women in Comp. Science (\href{http://awics.cs.tamu.edu/}{AWICS})}{Texas A\&M University}{}{}

	\cventry{2002-2004}
		{Teaching Assistant}{\href{http://www.wou.edu/las/natsci_math/math/}{Department of Mathematics}}{Western Oregon University}{}{}
		%{Assisted a Math professor in grading, lecturing.}

	\cventry{2002-2004}
		{Tutor}{\href{http://www.wou.edu/las/cs/index.php}{Dept. of Computer Science} and \href{http://www.wou.edu/las/natsci_math/math/}{Dept. of Mathematics}}{Western Oregon University}{}{}
		%{Tutored for various Computer Science and Math classes.}

	\cventry{2001-2004}{Youth Mentor}{HOST Youth and Family Program}{Northwest Human Services}{}{}
