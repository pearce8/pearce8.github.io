%!TEX root = ../olga-cv.tex
%----------------------------------------------------------------------------------
%            Teaching
%----------------------------------------------------------------------------------
\section{Teaching and Mentoring}
		
	\cventry{2021-present}{Teaching}{\href{https://engineering.tamu.edu/cse/index.html}{Texas A\&M University, Department of Computer Science and Engineering}}{}{}{
		\begin{itemize}
		\item (Fall 2022) CSCE 435 Parallel Computing, 95 students		
		\item (Fall 2021) CSCE 435 Parallel Computing, 60 students
		\end{itemize}
	}
	\cventry{2020-present}{Tutorials}{}{}{}{
		\begin{itemize}
		\item (May 2022) Exascale Computing Project (ECP), half-day tutorial, 
			\href{https://www.exascaleproject.org/event/perfanalysis/}
				{``Automated application performance analysis with Caliper, Spot, and Hatchet"}
		\item (Nov 2021) ACM/IEEE Int'l Conf. for High Performance Computing, Networking, Storage, and Analysis (SC), 
			full-day tutorial, \href{https://sc21.supercomputing.org/presentation/?id=tut128&sess=sess206}
				{``User-centric Automated Performance Analysis of Hybrid Parallel Programs"}
		\item (Mar 2021) Exascale Computing Project (ECP), full-day tutorial, 
			\href{https://www.exascaleproject.org/event/perfanalysis/}
				{``Automating application performance analysis with Caliper, Spot, and Hatchet"}
		\item (July 2020) Weapons Simulation Codes (WSC), 2 tutorials, ``Performance Analysis With Hatchet"
		\end{itemize}
	}
	\cventry{2015-present}{Student Research Mentoring}{Lawrence Livermore National Laboratory}{}{}{
		\begin{itemize}
		\item (Summer 2022) Garrett Hooten, CS major (Bachelors), University of Tennessee Chattanooga.
		  Worked on instrumenting ExaMPI with Caliper and performance analysis.			
		\item (Spring 2022 - Present) Michael McKinsey, CS major (Masters), Texas A\&M University.
		  Working on multi-dimensional data composition for \href{https://thicket.readthedocs.io/en/latest/}{Thicket}.
		\item (Spring 2022 - Present) Vanessa Lama, CS major (Masters), University of Tennessee.
		  Working on metadata for \href{https://thicket.readthedocs.io/en/latest/}{Thicket}.
	         \item (Spring 2022 - Present) Treece Burgess, CS major (Masters), University of Tennessee.
		  Working on statistics operations for \href{https://thicket.readthedocs.io/en/latest/}{Thicket}.  
		\item (Summer 2021 - Present) Daryl Hawkins, NUEN major (PhD), Texas A\&M University.
		  Working on novel transport algorithms.  Working on analyzing performance of MARBL on AWS.		  	  
		\item (Summer 2021) Gerald Collom, CS major (PhD), University of New Mexico.
		  Working on persistent and partitioned MPI for heterogeneous architectures.		
	        \item (Summer 2020 - Present) Connor Scully-Allison, CS major (PhD), University of Arizona.
	          Working on interactive visualization for \href{https://thicket.readthedocs.io/en/latest/}{Thicket}.
		  Worked on interactive visualization for \href{https://llnl-hatchet.readthedocs.io/en/latest/}{Hatchet}.
		  Worked on optimization of \href{https://llnl-hatchet.readthedocs.io/en/latest/}{Hatchet},
        		  a call-tree based tool for performance analysis of HPC applications.
	        \item (Spring 2020 - Present) Ian Lumsden, CS major (PhD), University of Tennessee.
		  Working on a query language for \href{https://llnl-hatchet.readthedocs.io/en/latest/}{Hatchet},
        		  a call-tree based tool for performance analysis of HPC applications.
	        \item (Summer 2020) Kawthar Shafie Khorassani, CS major (PhD), The Ohio State University.
		  Worked on GPUDirect Async Designs with MVAPICH2-GDR.
        		\item (Summer 2020) Katy Williams, CS major (PhD), University of Arizona.
		  Worked on interactive tree visualization for \href{https://llnl-hatchet.readthedocs.io/en/latest/}{Hatchet},
	          a call-tree based tool for performance analysis of HPC applications.
		\item (Fall 2019 - Present) Bengisu Elis, CS major (PhD), Technical University of Munich, Germany.
		  Working on research directions for MPI on heterogeneous architectures.
		\item (Fall 2019 - Present) Suraj Kesavan, CS major (PhD), University of California Davis.
		    Working on research directions for performance visualization.	          
		\item (Summer 2018 and 2019) David Eberius, CS major (PhD), University of Tennessee.
		  Worked on porting \href{http://www.exmatex.org/comd.html}{CoMD}, an ExMatEx proxy application to
		  \href{https://github.com/LLNL/RAJA}{RAJA}, and optimizing communication on GPUs.
		  Worked on load balancing metrics for heterogeneous architectures.
		\item (Summer 2018) Kewen Meng, CS major (PhD), University of Oregon.
		  Worked on porting \href{http://www.exmatex.org/comd.html}{CoMD}, an ExMatEx proxy application to
		  \href{https://github.com/LLNL/RAJA}{RAJA},
		  and optimizing performance for the CUDA RAJA backend.
		\item (Summer 2016) Hadia Ahmed, CS major (PhD), University of Alabama at Birmingham.
		  Studied performance impact of initial load imbalance and dynamic imbalance in the modified version of
		  \href{http://www.exmatex.org/comd.html}{CoMD}, an ExMatEx proxy application.
		\item (Summer 2015) Rasmus Larsen, CS major (MS), University of Copenhagen. %   \hfill {\em Summer 2015}\\
		  Enabled overdecomposition and dynamic work redistribution in
		  \href{http://www.exmatex.org/comd.html}{CoMD}, an ExMatEx proxy application.
		  Implemented a mechanism for introducing initial load imbalance and
		  varying the imbalance throughout the simulation
		  to facilitate exploration of various load balancing schemes.
		\end{itemize}
	}
	\cventry{2007}{Student Team Mentoring}{Cluster Challenge}{ACM/IEEE Supercomputing Conference (SC)}{}{
		\begin{itemize}
		\item Mentored teams participating in the challenge, provided technical expertise in parallel performance and
			parallel applications ported by the teams to their clusters during the challenge.
		\end{itemize}
		}
	\cventry{2005-2009}{Student Research Mentoring}{Texas A\&M University}{}{}{
		\begin{itemize}
		\item Jeremy Vu, CSE major, Texas A\&M University. %   \hfill {\em Spring 2009}\\
			Worked on parallel algorithms in STAPL.
		\item Harshvardhan, CS major, %\hfill {\em Fall 2005 -- Spring 2007}\\
			%University Undergraduate Research Fellow,
			Texas A\&M. %, graduated December 2008.
			Worked on STAPL Graph, Dynamic Graph Composition Library.
		\item Kelli Bacon, CRA-W DMP Program. CE major, Gonzaga U. %  \hfill {\em Summer 2006} \\
			Dynamic Graph Composition Library.
		\item Saransh Mittal, CSE major, %   \hfill {\em Summer 2006} \\
			Indian Inst. of Technology, Bombay.
			Dynamic Graph Composition Library.
		\item Anna Tikhonova, CRA-W DMP Program. CS major, U. of San Francisco. %   \hfill {\em Summer 2005} \\
			%Currently a Ph.D. student at University of California - Davis.\\
			Parallel algorithms in STAPL.
		\end{itemize}
		}
	\cventry{2006}{Fellow}{Graduate Teaching Academy}{Texas A\&M University}{}{}

	\cventry{2006}{Student Mentoring Award}{Aggie Women in Comp. Science ({AWICS})}{Texas A\&M University}{}{}

%	\cventry{2002-2004}
%		{Teaching Assistant}{\href{http://www.wou.edu/las/natsci_math/math/}{Department of Mathematics}}{Western Oregon University}{}{}
%		%{Assisted a Math professor in grading, lecturing.}
%
%	\cventry{2002-2004}
%		{Tutor}{\href{http://www.wou.edu/las/cs/index.php}{Dept. of Computer Science} and \href{http://www.wou.edu/las/natsci_math/math/}{Dept. of Mathematics}}{Western Oregon University}{}{}
%		%{Tutored for various Computer Science and Math classes.}
%
%	\cventry{2001-2004}{Youth Mentor}{HOST Youth and Family Program}{Northwest Human Services}{}{}
