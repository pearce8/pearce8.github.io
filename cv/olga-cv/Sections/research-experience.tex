%!TEX root = ../olga-cv.tex
%----------------------------------------------------------------------------------
%        Research Experience
%----------------------------------------------------------------------------------
\section{Experience}

	\cventry{2021-present}
		{LLNL}{Principal Member of Technical Staff (MTS 4)}{Center for Applied Scientific Computing}{}
		{\begin{itemize}
		      \item %(2021-present) 
		      	Lead: Benchmarking for procurement of LLNL's future Advanced Technology Systems (ATS).  
		      \item Lead:  \href{https://github.com/LLNL/benchpark}{Benchpark}, 
		      	an open collaborative repository for reproducible specifications of HPC benchmarks.
		      \item Lead: Performance Analysis and Visualization at Exascale (PAVE) project.
		 %\item (2021-present) (879k/year) Principal Investigator for Performance Analysis and Visualization at Exascale
		  %\item (2019-present) (545k/year) Principal Investigator for Performance Analysis and Visualization at Exascale
		      \item Lead: \href{https://thicket.readthedocs.io/en/latest/}{Thicket}, open source project for
		      	exploratory data analysis (EDA) for multi-experiment, multi-architecture, multi-tool parallel performance data.
		      \item Lead: Benchmarking for Acceptance for the 
		      	\href{https://www.llnl.gov/news/llnl-and-hpe-partner-amd-el-capitan-projected-worlds-fastest-supercomputer}{El Capitan}
			procurement.
%\item (2018-present) LLNL application point of contact for CORAL1 and CORAL2 Tools Working Groups~\cite{pearce:Coral2Mar20}.
%\item (2018-present) LLNL application point of contact for CORAL1 and CORAL2 Messaging Working              
%		                   Groups~\cite{pearce:sc19,pearce:ibm19,pearce:summitApr20,pearce:summitDec19,pearce:summitAug19}.
%\item (2020-2022) Predictive Science Academic Alliance Program (\href{https://psaap.llnl.gov}{PSAAP}) Tri-Lab Strategy Team, 
%Center for Understandable, Performant Exascale Communication Systems, University of New Mexico, 
%University of Tennessee at Chattanooga, and University of Alabama at Birmingham.
		 \end{itemize}
		}
\cventry{2021-present}
		{Texas A\&M University, Dept of Computer Sci \& Engineering}
		{Associate Professor of Practice}
		{Teaching CSCE 435 Parallel Computing}{256 students over 3 years}{}
		%\begin{itemize}
		  %\item (Fall 2023) Teaching CSCE 435 Parallel Computing, 100 students
		  %\item (Fall 2022) Teaching CSCE 435 Parallel Computing, 95 students
		  %\item (Fall 2021) Teaching CSCE 435 Parallel Computing, 60 students
		%\end{itemize}	
\cventry{2014-2021}
	{LLNL}{Senior Member of Technical Staff (MTS 3)}{Center for Applied Scientific Computing}{}
		{\begin{itemize}
		      \item Lead: Performance Analysis and Visualization at Exascale (PAVE) project.
		      \item Lead: \href{https://llnl-hatchet.readthedocs.io/en/latest/}{Hatchet}, a call-tree based tool for performance analysis. 
		\item Developed techniques \& tools for analyzing \& optimizing performance 
			of large scale multiphysics simulations %~\cite{pearce:N210808} 
			on homogeneous \& heterogeneous architectures,
	                   e.g., \href{https://software.llnl.gov/Caliper/index.html}{Caliper}.
                   %~\cite{pearce:isc21,boehme:sc16,pearce:necdc18,pearce:salishan19,pearce:cascWIP19,pearce:gtc18,pearce:jowog18}.
		\item Developed techniques and tools for load balancing scientific applications.%~\cite{pearce:fgcs18,pearce:sc16pmbs}.
		\item Developed parallel programming models and tools, including \href{https://raja.readthedocs.io/en/main/}{RAJA}.
          %~\cite{beckingsale:ipdps17,pearce:p3hpc19,pearce:pc19,pearce:p2s218,pearce:gtc19,pearce:jowog19,pearce:coepp17,trilabL2:15}.
		\item Developed communication techniques for heterogeneous architectures,%~\cite{pearce:parco20}.
          	exploratory benchmark \href{https://github.com/LLNL/Comb}{Comb}.		
			%~\cite{pearce:summitApr20,pearce:summitDec19,pearce:summitAug19}.
          	\item Member of \href{https://github.com/mpiwg-hybrid/hybrid-issues}{MPI Hybrid \& Accelerator Working Group}.
		\end{itemize}}