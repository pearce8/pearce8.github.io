%!TEX root = ../olga-cv.tex
%----------------------------------------------------------------------------------
%        Research Experience
%----------------------------------------------------------------------------------
\section{Research Experience}
	\cventry{2014-present}
		{\href{http://www.llnl.gov}{Lawrence Livermore National Laboratory}}{Computer Scientist}{}{}
		{\begin{itemize}
		  \item (2019-present) (545k/year) Principal Investigator for Performance Analysis and Visualization at Exascale
            (\href{https://computing.llnl.gov/projects/pave-performance-analysis-visualization-exascale}{PAVE}) Project,
            which includes the \href{https://hatchet.readthedocs.io/en/latest/}{Hatchet} project, a call-tree based tool
            for performance analysis of HPC applications~\cite{pearce:ProToolsSC20,pearce:vpa18,pearce:jowog20}.
		\item (2018-present) LLNL application point of contact for CORAL1 and CORAL2 Tools Working Groups~\cite{pearce:Coral2Mar20}.
		\item (2018-present) LLNL application point of contact for CORAL1 and CORAL2 Messaging Working Groups~\cite{pearce:sc19,pearce:ibm19,pearce:summitApr20,pearce:summitDec19,pearce:summitAug19}.
		\item Developing techniques and tools for analyzing and optimizing performance of large scale multiphysics simulations on homogeneous and heterogeneous architectures,
          including \href{https://software.llnl.gov/Caliper/index.html}{Caliper}~\cite{boehme:sc16,pearce:salishan19,pearce:cascWIP19,pearce:necdc18,pearce:gtc18,pearce:jowog18}.
		\item Developing techniques and tools for load balancing scientific applications~\cite{pearce:fgcs18,pearce:sc16pmbs}.
		\item Developing parallel programming models and tools, including \href{https://raja.readthedocs.io/en/main/}{RAJA}
          ~\cite{beckingsale:ipdps17,pearce:p3hpc19,pearce:pc19,pearce:p2s218,pearce:gtc19,pearce:jowog19,pearce:coepp17,trilabL2:15}.
		\item Developing communication techniques for heterogeneous architectures~\cite{pearce:parco20}.
          Exploratory benchmark \href{https://github.com/LLNL/Comb}{Comb}~\cite{pearce:summitApr20,pearce:summitDec19,pearce:summitAug19}.
          Member of \href{https://github.com/mpiwg-hybrid/hybrid-issues}{MPI Hybrid \& Accelerator Working Group}.
		\end{itemize}}

	\cventry{2009-2014}
		{\href{http://www.llnl.gov}{Lawrence Livermore National Laboratory}}{Lawrence Scholar}{}{}
		%{Developing a framework for asynchronous scalable correction of application load imbalance}
		{\begin{itemize}
            \item Developed a model enabling the comparison of and selection among
                  balancing algorithms for a specific state of a simulation~\cite{pearce:ics12}.
            \item Developed an accurate and fast method to evaluate and balance the load in N-Body simulations
                  with highly non-uniform density, based on adaptive sampling~\cite{pearce:ics14}.
            \item Developend a framework for decoupling and offloading the load balance computation,
                  enabling lazy load balancing~\cite{pearce:ipdps16}.
			      %\item Developed an application-independent load model and metrics;
			%\item Developed an algorithmic methodology for deciding {\em when} and {\em how} to correct application imbalance;
			%\item Developed a fast and accurate method of assigning and load balancing interactions in N-body applications;
			%\item Developed a framework for decoupling load balance algorithm resources from the application;
			%\item Developed a framework for offloading \& overlapping load balance computation with application execution;
			%\item Applied work to several large physics applications at LLNL.
		\end{itemize}}

	\cventry{2004-2009}
		{\href{http://www.cse.tamu.edu/}{Texas A\&M University, Department of Computer Science}}{Research Assistant}{}{}
		{\begin{itemize}
			\item Worked on distributed data structures in the Standard Template Adaptive Parallel Library (STAPL),
			  a parallel superset of the C++ Standard Template Library~\cite{tanase:ppopp11,buss:systor10,tanase:lcpc09}.
              %Implemented and optimized a parallel vector.
			\item Designed and implemented a set of parallel algorithms for STAPL.
				Assisted in development of the adaptive algorithm selection framework for STAPL~\cite{thomas:ppopp05}.
			\item Explored the idea of lock-free containers for multi-threaded programming.
			%\item Working on a general scheduler and partitioner for STAPL, which will
			%be used to schedule and load-balance various applications using STAPL.
			\item Worked on a Dynamic Graph Composition Library for updating graph metrics of large, dynamic graphs.
		\end{itemize}}

	\cventry{May-Oct 2007,\\July-Aug 2008}
		{\href{http://www.llnl.gov}{Lawrence Livermore National Laboratory}}{Research Intern}{}{}
		{Worked on load balancing a multi-physics simulation via domain repartitioning
		and domain replication.  Devised techniques to cheaply estimate processor loads and decide
		when a load imbalance needs to be corrected.  Incorporated multi-constraint
		repartitioning (data parallelism) and domain replication (task parallelism)
		as a way to correct imbalance.
		}

	\cventry{Summer 2003,\\Summer 2004}
		{\href{http://www.cse.tamu.edu/}{Texas A\&M University, Department of Computer Science}}{Undergraduate Research Intern}{}{}
		{Designed and implemented a parallel array data structure and multiple
		parallel algorithms for the Standard Template Adaptive Parallel Library
		(STAPL), a parallel library for C++, superset of STL.}
