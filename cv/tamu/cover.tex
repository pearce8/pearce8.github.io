\documentclass[11pt]{article}
\begin{document}

Dear Search Committee,

I am writing to apply to the position of Professor of Practice in the Department of Computer Science and Engineering at the Texas A\&M University.  I am currently a Computer Scientist in the Center for Applied Computer Science (CASC) at the Lawrence Livermore National University (LLNL).  Previously, I earned my Ph.D. with Dr. Nancy Amato at TAMU.

My vision is to create a learning environment that resembles the multidisciplinary teams across the nation's finest industrial and research institutions,
fully engaging the students' imagination and ability to solve problems individually and as a team.

My experience to date has laid the foundation for these goals.  I earned my doctorate at TAMU under Dr. Nancy Amato, where I developed parallel algorithms and data structures.
On the basis of this work, I received a fellowship at LLNL to finish writing my dissertation, under the supervision of Dr. Bronis de Supinski.
During the fellowship, I developed load balancing algorithms for full scale simulations, along with models to make decisions on when and how to load balance.

Upon graduation, I started a staff position in CASC, and established myself as an expert in programming models, interprocess communication, and performance analysis and tools.
Working with multidisciplinary multi-physics simulation teams, I leveraged my expertise to help gain 15x speedup by leveraging heterogeneous architectures.
To achieve it, we developed the RAJA programming model, enabling a single source application to run on CPUs or GPUs.
We stood up a toolchain for application performance measurement (Caliper), visualization (Spot), and programmatic analysis (Hatchet).
As PI of the Performance Analysis and Visualization Project, I lead a team of researchers working with four professors across the nation
to develop several tools (including Hatchet).

While Sierra (currently \#3 on Top500 list) was being stood up at LLNL, I worked closely with IBM and NVIDIA on the programming models,
communication mechanisms, and tools for the machine to make it possible for LLNL teams to use it to its fullest potential.
While that work is on-going, we are also in progress co-designing the next supercomputer for LLNL, El Capitan, with HPE-Cray and AMD.
For El Capitan, I am driving the collaborations on the communication and tools fronts.  To that goal,
I am also active in the research community, including participating in the MPI Forum and the premier conferences in the field.

I believe my national lab experience would contribute to the diverse research environment in the CSE department at TAMU, and I would be excited to form collaborations with several groups thoughout the University.

%Coming to America to pursue my aspiration to advance science has

%As an immigrant, I came to America to pursue the American dream of achieving my aspirations to advance science.

%Like many American immigrants, I chose to come here to pursue the American dream of following my aspirations to advance science.




\end{document}
