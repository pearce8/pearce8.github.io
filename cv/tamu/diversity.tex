%!TEX TS-program = multibib
\documentclass[10pt,letterpaper]{moderncv/moderncv}
\usepackage{wrapfig}

%----------------------------------------------------------------------------------
% moderncv Setup
%----------------------------------------------------------------------------------
\moderncvstyle{classic}                      % style options are 'casual' (default), 'classic', 'oldstyle' and 'banking'
\moderncvcolor{purple}                         % color options 'blue' (default), 'orange', 'green', 'red', 'purple', 'grey' and 'black'
\usepackage[utf8]{inputenc}                  % replace by the encoding you are using
%\renewcommand{\familydefault}{\sfdefault}   % to set the default font; use '\sfdefault' for the default sans serif font, '\rmdefault' for the default roman one, or any tex font name
\nopagenumbers{}                             % uncomment to suppress automatic page numbering for CVs longer than one page

%----------------------------------------------------------------------------------
% Geometry and margins.
%----------------------------------------------------------------------------------
\usepackage[scale=0.8,top=1in,bottom=.75in]{geometry}
\setlength{\hintscolumnwidth}{2.8cm}                         % if you want to change the width of the column with the dates
%\AtBeginDocument{\setlength{\maketitlenamewidth}{6cm}}  % only for the classic theme, - to change width of name placeholder (to leave more space for your address details)
\AtBeginDocument{\recomputelengths}                      % required when changes are made to page layout lengths

%----------------------------------------------------------------------------------
% Set up hyperref custom link colors.
%----------------------------------------------------------------------------------
%\usepackage{hyperref}
\usepackage{xcolor}                          % Use xcolor so I can steal colors from moderncv
\colorlet{mylinkcolor}{color1!80!black}      % Links are slightly darker than color1
%\hypersetup{
%    colorlinks=true,%
%    citecolor=black,%
%    filecolor=black,%
%    linkcolor=mylinkcolor,%
%    urlcolor=mylinkcolor,%
%    pdfborder=0 0 0%
%}
\urlstyle{sf}    % Make URLs match rest of text.

%----------------------------------------------------------------------------------
% Personal data.
%----------------------------------------------------------------------------------
\firstname{Olga}
\familyname{Pearce}
%\title{Resum\u00E9 title (optional)}                       % optional

%%%%%%%%% LLNL address
\address{Lawrence Livermore National Laboratory}{Livermore, CA, 94550}    % optional
%\phone{925.422.0436}                                   % optional
%\fax{fax (optional)}                                  % optional
\email{olga@llnl.gov}                             % optional
\extrainfo{\href{http://people.llnl.gov/olga}{\url{http://people.llnl.gov/olga}}}

%%%%%%%%%%%% TAMU address
%\address{Parasol Lab, Dept. of Computer Science and Engineering\\Texas A\&M University,}{College Station, TX 77843-3112}
%%\phone{925.422.0436}                                   % optional
%%\fax{fax (optional)}                                  % optional
%\email{olga@cs.tamu.edu}                             % optional
%\extrainfo{\href{http://parasol.tamu.edu/people/olga}{\url{http://parasol.tamu.edu/people/olga}}}
%%%%\quote{Some quote (optional)}                         % optional
%%%%\photo[64pt]{picture}                                 % optional, '64pt' is the height the picture must be resized to and
%                                                       %'picture' is the name of the picture file

%----------------------------------------------------------------------------------
% Custom commands.
%----------------------------------------------------------------------------------
% command and color used in this document, independently from moderncv
\definecolor{see}{rgb}{0.5,0.5,0.5}% for web links
\newcommand{\see}[1]{\hfill{\itshape\color{see}\footnotesize{}see #1}}

%----------------------------------------------------------------------------------
% Bibliography Customizations.
%----------------------------------------------------------------------------------
% Put labels next to best posters, based on bibliography order.
% NOTE: this is a little nasty; numbers need to be updated every time the pubs change.
\makeatletter
\renewcommand{\bibliographyitemlabel}{
    \ifthenelse{\value{enumiv}=160\or%
                \value{enumiv}=150}{\color{color1}\textbf{Best Poster} \color{black}}{}
    {\@biblabel{\arabic{enumiv}}}%
}
\makeatother



\begin{document}
\recipient{\large Diversity Statement}{Olga Pearce, Ph.D.}
\date{January 20, 2021}
\opening{}
%\enclosure[Attached]{curriculum vit\ae{}}          % use an optional argument to use a string other than "Enclosure", or redefine \enclname
\makelettertitle%\justifying
\vspace{-.75cm}

Walking into my first Computer Science course in college and being asked if I am lost
was one of the jarring experiences that could have deterred a younger me from pursuing my passion.
Class after class, I had to prove that I could ``do it'' to my classmates and my professors alike.
But soon, a professor I regarded highly suggested I apply for the Distributed Mentor Program,
a CRA-W program for undergraduates in teaching schools to gain summer research experience at R1 schools.
My professor insisted I apply despite my hesitation and wrote a letter of support --
and as a result that summer I was given the opportunity to work in Prof. Nancy Amato's research group,
in a challenging and fascinating research program that launched my research career.

In graduate school, I quickly found myself a part of the Aggie Women in Computer Science (AWICS), volunteering as a mentoring officer
and eventually the president of the group for several years.  In AWICS, I formalized my understanding of why
mentoring is so crucial at every stage of one's career, and how we can all mentor those just a few years
fewer into theirs.

As a graduate student, I was a technical mentor to five undergraduate students, and mentored several others.
As my professional role grew, I mentored and supervised a dozen female Ph.D. students, post-docs,
and early career scientists as staff at the Lawrence Livermore National Laboratory, and dozens of undergraduate students as a member of
Supercomputing committee on student-related programs (Student Cluster Challenge, Broader Engagement,
and Vice Chair of the Student program).  I firmly believe that presenting a successful example in the field,
being available for career questions, as well as mentoring, can help many female computer scientists overcome the superficial barriers of a mostly male field.

In my work with students as TAMU, I would like to pursue the following objectives:
\begin{enumerate}
\item \emph{develop new talent:} help attract, retain, and motivate a diverse set of students with talent and skills necessary to learn and advance the cutting edge research in computer science;
\item \emph{motivate the team:} support my students and research team in setting and meeting high academic objectives;
\item \emph{model personal excellence:} mentor and coach others in order to strengthen the CSE Department and TAMU, act with integrity under all circumstances, and model work/life balance.
\end{enumerate}


\end{document}
