\documentclass[11pt]{article}
\begin{document}

Walking into my first Computer Science course in college and being asked if I am lost was one of the jarring experiences that could have deterred a younger me from pursuing my passion.  Class after class, I had to prove that I could ``do it''
to my classmates and my professors alike.  Nearly two decades and several degrees later, I still have to push through the initial shock of a woman wielding significant technical chops in the mostly male field.

While I have learned how to mostly gracefully get through these early impression issues, I recognize how challenging these may be to young students entering the field, and I am passionate about helping them get past these roadblocks and into rewarding, fascinating careers computer scientists can choose to pursue.  As my professional role grew, I have mentored and supervised a dozen female Ph.D. students, post-docs, and early career scientists as staff at LLNL, and dozens of undergraduate students as a member of Supercomputing committee on student-related programs (Student Cluster Challenge, Broader Engagement, and Vice Chair of the Student program).  I firmly believe that presenting a successful example in the field, being available for career questions, as well as mentoring, can help many female computer scientists overcome the superficial barriers of a mostly male field.

In my work with students as TAMU, I would like to pursue the following objectives:

develop new talent: help attract, retain, and motivate a diverse set of students with talent and skills necessary to learn and advance the cutting edge research in computer science;
motivate the team: support my students and research team in setting and meeting high academic objectives;
model personal excellence: mentor and coach others in order to strengthen the CSE Department and TAMU, act with integrity under all circumstances, and model work/life balance~



\end{document}
