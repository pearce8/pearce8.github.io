\documentclass[11pt]{article}
\begin{document}

Walking into my first Computer Science course in college and being asked if I am lost
was one of the jarring experiences that could have deterred a younger me from pursuing my passion.
Class after class, I had to prove that I could ``do it'' to my classmates and my professors alike.
But soon, a professor I looked up to most suggested I applied for the Distributed Mentor Program --
a CRA-W program for undergraduates in teaching schools to gain summer research experience at R1 schools.
My professor insisted I applied despite my hesitation, and wrote me a letter of recommendation for the program --
and that summer I found myself in a challenging and fascinating research program that launched my research career.
In graduate school, I quickly found myself a part of the Aggie Women in Computer Science (AWICS), volunteering as a mentoring officer
and eventually the president of the group for several years.  In AWICS, I formalized my understanding of why
mentoring is so crucial at every stage of one's career, and how we can all mentor those just a few years
fewer into theirs.

As a graduate student, I was a technical mentor to five undergraduate student, and mentored several others.
As my professional role grew, I mentored and supervised a dozen female Ph.D. students, post-docs,
and early career scientists as staff at LLNL, and dozens of undergraduate students as a member of
Supercomputing committee on student-related programs (Student Cluster Challenge, Broader Engagement,
and Vice Chair of the Student program).  I firmly believe that presenting a successful example in the field,
being available for career questions, as well as mentoring, can help many female computer scientists overcome the superficial barriers of a mostly male field.

In my work with students as TAMU, I would like to pursue the following objectives:
\begin{enumerate}
\item {\bf develop new talent:} help attract, retain, and motivate a diverse set of students with talent and skills necessary to learn and advance the cutting edge research in computer science;
\item {\bf motivate the team:} support my students and research team in setting and meeting high academic objectives;
\item {\bf model personal excellence:} mentor and coach others in order to strengthen the CSE Department and TAMU, act with integrity under all circumstances, and model work/life balance
\end{enumerate}


\end{document}
