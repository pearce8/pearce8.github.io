\documentclass[11pt]{article}
\begin{document}

It is my pleasure to recommend Olga Pearce for the Professor of Practice position at Texas A\&M University.

. . . about the recommender here ...

I have known Olga through the conference committee work for nearly a decade, and have worked closely with her over the past year on the Hatchet project.

Olga is the PI for the Hatchet project, a performance analysis tool that enables users to programmatically compare multiple performance profiles, and write their own analysis.
Hatchet is a young two year old project, but has already attracted many users.
As a tools researcher and a performance analysis expert working with many large scale parallel simulation codes,
Olga was instrumental in developing the initial idea for Hatchet to fill the need for programmatic performance analysis.

While it was important to develop a tool that would work with existing profiling and tracing tools,
Olga early on recongnized the importance of interfacing with the in-house profiling tool called Caliper,
which Olga has been helping integrate into several of LLNL's simulation codes.
Olga was instrumental in integrating Hatchet into Spot,
a web interface code developers use to view the performance data Caliper collects about
executions of their simulations.
Because code developers rely on Olga as the tools expert,
they would quickly call her to find out the answer to
``what does this button do?'', at which point she would recruit them
as early users.

As feature requests poured in, Olga recruited several students
to help the otherwise small Hatchet team.
At this point, Hatchet has two LLNL staff (counting Olga) working on it,
three university professors (from different institutions)
with one or two students each.
The team now includes experts in simulation optimization, performance tools,
and visualization, enabling creative solutions that are
rapidly delivered to the very engaged users of the tool.

While the early adoption of Hatchet was by word of mouth,
Olga has since lead several Hatchet tutorials at LLNL.
Olga proactively talks to the current and prospective Hatchet users,
bringing back feature requests and ideas to the Hatchet team.


%
%

In the wider Computer Science community, Olga is very active in working with students,
over the years serving on the Cluster Challenge, Broader Engagement, and Student committees at
the ACM/IEEE Supercomputing Conference,
and the Ph.D. forum at the International Conference on Supercomputing.
%
Olga's technical expertise is widely recognized, and Olga has been invited to and serves on
prominent technical committees in her field, such as
the ACM/IEEE Supercomputing Conference,
IEEE International Parallel \& Distributed Processing Symposium,
EuroMPI, IEEE International Conference on Cluster Computing,
International Conference on Parallel Processing,
IEEE/ACM International Conference on Cluster, Cloud, \& Grid Computing,
International Conference on Supercomputing,
as well as a reviewer for several journals.
%
Additionally, Olga co-chairs the
Salishan Conference on High Speed Computing,
guiding the direction of this premier invitation-only conference
on computer architecture, languages, and algorithms organized by the tri-labs (LLNL, LANL, Sandia).






%intro paragraph. first sentence: it is my pleasure to recommend xxx for the position xxx at xxx. (the recommender will edit the sentence but it's good to have a starter) Then the rest of the paragraph introduces the writer and their qualifications. Why are they qualified to recommend you and in what context. How long have you worked together or known each other This is hard for you to draft. I would just do my best and let the writer fill in the rest


several paragraphs describing the work you have done and its impact. It is good to give examples like papers or software. Think of two or three things that you think academics would think is important: publication record, teaching/mentoring experience, research impact or software. Then describe achievements in those areas. Call out specific papers and their topics that you think are good, etc.Then usually a sentence that you can follow up if you have any questions



\end{document}
